\section{Related Work}
The idea of using models to characterize normal behavior is that has been tried across a wide spectrum of systems ranging from industrial control plants \cite{}, capturing the physics of sensors \cite{}, medical devices \cite{}, and other control systems \cite{}. The equivalent of a model in control theory is called \textit{state estimation}. The central idea is to characterize the variations in key parameters as well as the noise that is inherent in the physical environment and the measurement. Models typically capture a universal law in the domain of application. For instance, in industrial control plants, laws of fluid dynamics such as Bernoulli's equations can be used to capture the model. However, as the applications of models have varied wildly from one another, there has not been enough discussion to provoke the use of model-based anomaly detection in a general sense. \\

\textbf{Security \& Privacy in CPS}. Another domain to look for related work is in the security and privacy of CPS systems. An example is is verification of control code by an embedded system before it reaches the Programmable Logic Controller(PLC), Remote Terminal Unit or Intelligent Electronic Device(IED) \cite{}. This work came out in the context of the popular Stuxnet \cite{} security breach and triggered a bunch of other works in this space. Examples being security of embedded devices \cite{}, the automatic generation of malicious PLC payloads \cite{}, medical device security \cite{}, vulnerability analysis of vehicles \cite{}, and of automated meter readings \cite{}. There is work on CPS privacy such as smart grids \cite{}, vehicular location monitoring \cite{}, and location privacy \cite{}. These are related but complementary to the problem that we are describing in this paper. 

Instead, in this work,we are going to be focusing on using measurements of the physical world via sensors to build up an indicator of attacks. This work is inspired by the work on false sensor measurements \cite{}, or false control signals such as manipulating vehicle platoons \cite{}, manipulating demand-response systems \cite{}. The question we ask is this: \textit{How do I determine when something is wrong with sensors or actuators?}. Intrusion detection is a problem that is closely related to this. A classic paper which considers intrusion detection in industrial control networks is Cheung et al. \cite{}. It is noted in this paper that the technique of network anomaly detection is more effective in bounded, finite control networks where network flows are more regular and stable compared to the traditional computer networks used in the internet.

\textbf{Secure System Identification}One of the main areas of research in the CPS community is to find efficiently the subset of sensors that are sending false information \cite{}. These systems are assumed to satify the equations for Linear, Time-invariant systems which has been noted earlier. The main idea here is to solve a combinatorial optimization problem to find a subset of sensors, which are assumed to be safe, to generate adequate state estimations. Shoukry et. al \cite{} proposed a search algorithm based on Satisfiability Modulo Theory(SMT) to speed up the search of possible sensor sets and extended this to systems subject to random noise \cite{}. In particular, it was noted that the number of sensors used to monitor the process has to be atleast twice the number of sensors under attack. In this paper, the authors  assume that the controllers and actuators can be trusted. Furthermore, there is a degree of hardware redundancy assumed in this paper which might not be always true in practical systems

\textbf{Attacks on UAVs} Attacks on UAVs do not surpise anymore. In \cite{}, the internal vehicular networks are infiltrated by subverting the CAN bus. CAN bus, being a broadcast-based protocol is shown to be subverted by an adversary with a laptop having a wireless card. In \cite{}, the authors exploit a car tire pressure sensing system, which uses RF wireless motes. Kerns et. al \cite{} consider how the Global Positioning System (GPS) signals can be used to take over UAVs. The attacker generates a counterfeit GPS signal and sends it to the GPS antenna of the UAV. This replaces the real GPS reading with a fake reading. The paper also proposes a detection strategy by modeling the UAV's state and using a \textit{residual test} (a system that takes in sensor readings outputs a signal called the residual as per a mathematical formulation). This paper is interesting and shows 2 attacks -- one where the attacker is detected, and another where the attacker manages to keep all the residuals below the threshold while still steering the aircraft to their location. Optical sensor input spoofing \cite{} involves obtaining an implicit control channel by tricking optical flow sensors with a fake ground plane. In \cite{}, the authors propose using a gyroscopic sensor attack with intentional acoustic noise to crash drones. Furthermore, the authors of \cite{} comprimise accelerometers by injecting acoustic noise in a more controlled manner, as a more advanced form of attack. Anti-lock Braking System(ABS) attack \cite{} involves injecting magnetic field to spoof the wheel speed sensor. In \cite{}, an attacker with an antenna and malicious ground station can compromise a benign UAV by sending malicious packets. In \cite{}, there are attacks on camera-based ground vehicles by relaying and spoofing signals.


\textbf{Detecting Attacks on UAVs} Detecting attacks on UAVs has been explored by multiple angles. The approaches tried out by the community can be classified into four buckets. Note that all of these approaches can be broadly called as model-based approach.

\begin{enumerate}
	\item Signature-based detection
	\item Machine Learning (Classification)
	\item System redundancy
	\item Formal specification
\end{enumerate} 
 
Signature-based detection consists of a monitoring mechanism and compares it with pre-determined attack patterns known as attack signatures. This approach is popular among anti-virus corporations as well and relies on maintaining a database of signatures, which might curtail the usefulness of this approach. Machine learning-based approach monitors abnormal behaviors using a technique such as trained neural network or deep learning. The normal behavior is defined by supervised and unsupervised methods in the training phase. The disadvantage of this approach is the large training dataset that is needed to cover all the corner cases. Although unsupervised learning systems remove the need for labeled data, the issue of high false positives remain. Redundancy-based techniques duplicate key system components (such as the controller) and cross-check their states/outputs at runtime to detect intrusions/attacks/anomalies. The redundancy can be in hardware or it can be in software or a combination of both. However, from an attacker's standpoint, having redundant sensors increases the attack surface while increasing the attackers effort by only a constant effort. In effect, this seems like the wrong game to play with the attackers. Formal specification based attacks rely on program execution invariants to capture normal system operations. The end result is a state machine that can be used to detect anomalous program states and transitions to detect anomalies/attacks. A summary of attack detection methods used in UAVs is given in Table \cite{}  





