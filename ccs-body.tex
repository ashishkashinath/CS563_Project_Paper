\section{Introduction}
Cyber-Physical Systems(CPS) consists of an end-to-end architecture of computation, communication and control. These systems, which are typically computationally resource-constrained are deployed in processes which affect everyone of us such as industrial plants for treatment of water, nuclear power plants, surgical robots, unmanned aerial vehicles(UAVs) and so on. These systems involve an interaction between sensors, actuators, controllers and compute elements.

This mix of compute components such as CPUs, Memory and Registers as well control components enable the platform to interface with the environment by using sensors and actuators. The sensors sense the information from the environment, the controller uses this input to calculate the \textit{control output} which is sent to the actuators that are usually motors. In practice, the sensors and actuators could be separated from one another and connected by networks as in the case of industrial plants. However, in case of UAVs, the sensors and actuators can be physically connected via an interconnect such as the Serial bus or I2C.

Furthermore, in practice, we have multiple of these control systems working in unison, each exchanging information over network. Typically, these systems are also connected to the internet to permit online status monitoring and diagnosis for instance. From a security point of view, CPS systems therefore, deal with every issue that an off-the-shelf networked computing device does. Mitigating attacks in the cyber realm requires every fix that goes into a typical networked computing device. In addition, the attack surface of a CPS system is made larger due to the attacks possible on the controls environment of the Cyber-Physical system. These attacks are the focus of this paper, specifically the attacks on the sensors and actuators. Moreover, the controls attacks on networked CPS system can be classified into two categories:

\begin{itemize}
    \item \textit{Attacks on sensors}: Sensors are subsystems that perceive the environment as well as the progress of the critical parameters of the mission and formulate the state of the CPS. The sensors can be compromised by means of attacking the sensing mechanism itself or by compromising the communication between the sensor and the controller.
    
    \item \textit{Attacks on actuators}: Attacks on actuators cause \textit{immediate physical impact} on the device by means of physical instability that can lead to a crashed drone or \textit{take down}. Similar to the attack on sensors, the attacks on actuators can be carried out by attacking the sensing mechanism itself or the control signals between the controller and the actuators.
    
\end{itemize}

In this paper, we would be focusing on the attacks on sensors for ease of exposition. The main contributions of this paper are:

\begin{itemize}
    \item Control Theoretic Framework: We analyze the best control theoretic framework to model the vehicles and look at what is the best way to frame the attacker threat model and the implications of the model.
    \item Formulating the control theoretic attacks that are possible on UAVs and RVs. We study the impact of sensor and actuator attacks on the control theoretic model here.
    \item 
\end{itemize}

Furthermore, it does not matter for us whether the sensing mechanism is attacked or whether the communcation link is affected since we are focused on augmenting the control program(such as SLAM) running in the controller. In particular, this paper zeroes in on control-theoretic security to UAVs and Robotic Vehicles(RVs).


\section{Related Work}
The idea of using models to characterize normal behavior has been tried across a wide spectrum of systems ranging from industrial control plants \cite{wang2014srid}, capturing the physics of sensors \cite{shoukry2015pycra}, medical devices \cite{hei2013pipac}, and other control systems \cite{mclaughlin2013cps}. The equivalent of a model in control theory is called \textit{state estimation}. The central idea is to characterize the variations in key parameters as well as the noise that is inherent in the physical environment and the measurement. Models typically capture a universal law in the domain of application. For instance, in industrial control plants, laws of fluid dynamics such as Bernoulli's equations can be used to capture the model. However, as the applications of models have varied wildly from one another, there has not been enough discussion to provoke the use of model-based anomaly detection in a general sense. \\

\subsection{Security \& Privacy in CPS} Security and privacy of CPS systems is a well-researched area. An example is is verification of control code by an embedded system before it reaches the Programmable Logic Controller(PLC), Remote Terminal Unit or Intelligent Electronic Device(IED) \cite{mclaughlin2014trusted}. This work came out in the context of the popular Stuxnet \cite{falliere2011w32} security breach and triggered a bunch of other works in this space. Examples being security of embedded devices \cite{lemay2009cumulative}, the automatic generation of malicious PLC payloads \cite{mclaughlin2012sabot}, medical device security \cite{rushanan2014sok}, vulnerability analysis of vehicles \cite{checkoway2011comprehensive}, and of automated meter readings \cite{rouf2012neighborhood}. There is work on CPS privacy such as smart grids \cite{jawurek2012sok}, vehicular location monitoring \cite{hoh2008virtual}, and location privacy \cite{shokri2011quantifying}. These are related but complementary to the problem that we are describing in this paper. 

Instead, in this work,we are going to be focusing on using measurements of the physical world via sensors to build up an indicator of attacks. This work is inspired by the work on false sensor measurements \cite{liu2011false}, or false control signals such as manipulating vehicle platoons \cite{gerdes2013cps}, manipulating demand-response systems \cite{tan2013impact}. The question we ask is this: \textit{How do we raise a flag when something is wrong with sensors or actuators?}. Intrusion detection is a problem that is closely related to this. A classic paper which considers intrusion detection in industrial control networks is Cheung et al. \cite{cheung2007using}. It is noted in this paper that the technique of network anomaly detection is more effective in bounded, finite control networks where network flows are more regular and stable compared to the traditional computer networks used in the internet.

\subsubsection{Secure System Identification} One of the main areas of research in the CPS community is to find efficiently the subset of sensors that are sending false information \cite{chong2015observability}. These systems are assumed to satisfy the equations for Linear, Time-invariant systems which has been noted earlier. The main idea here is to solve a combinatorial optimization problem to find a subset of sensors, which are assumed to be safe, to generate adequate state estimations. Shoukry et. al \cite{shoukry2018smt} proposed a search algorithm based on Satisfiability Modulo Theory(SMT) to speed up the search of possible sensor sets and extended this to systems subject to random noise \cite{mishra2017secure}. In particular, it was noted that the number of sensors used to monitor the process has to be atleast twice the number of sensors under attack. In this paper, the authors  assume that the controllers and actuators can be trusted. Furthermore, there is a degree of hardware redundancy assumed in this paper which might not be always true in practical systems

\subsection{Attacks on UAVs} Attacks on UAVs do not surprise anymore. In \cite{checkoway2011comprehensive, koscher2010experimental}, the internal vehicular networks are infiltrated by subverting the CAN bus. CAN bus, being a broadcast-based protocol is shown to be subverted by an adversary with a laptop having a wireless card. In \cite{ishtiaq2010security}, the authors exploit a car tire pressure sensing system, which uses RF wireless motes. Kerns et. al \cite{kerns2014unmanned} consider how the Global Positioning System (GPS) signals can be used to take over UAVs. The attacker generates a counterfeit GPS signal and sends it to the GPS antenna of the UAV. This replaces the real GPS reading with a fake reading. The paper also proposes a detection strategy by modeling the UAV's state and using a \textit{residual test} (a system that takes in sensor readings outputs a signal called the residual as per a mathematical formulation). This paper is interesting and shows 2 attacks -- one where the attacker is detected, and another where the attacker manages to keep all the residuals below the threshold while still steering the aircraft to their location. Optical sensor input spoofing \cite{davidson2016controlling} involves obtaining an implicit control channel by tricking optical flow sensors with a fake ground plane. In \cite{son2015rocking}, the authors propose using a gyroscopic sensor attack with intentional acoustic noise to crash drones. Furthermore, the authors of \cite{trippel2017walnut} compromise accelerometers by injecting acoustic noise in a more controlled manner, as a more advanced form of attack. Anti-lock Braking System(ABS) attack \cite{shoukry2013non} involves injecting magnetic field to spoof the wheel speed sensor. In \cite{highnam2016uncrewed}, an attacker with an antenna and malicious ground station can compromise a benign UAV by sending malicious packets. In \cite{petit2015remote}, there are attacks on camera-based ground vehicles by relaying and spoofing signals.


\subsubsection{Detecting Attacks on UAVs} Detecting attacks on UAVs has been explored by multiple angles. The approaches tried out by the community can be classified into four buckets. Note that all of these approaches can be broadly called as \textit{Model-based} approaches.

\begin{enumerate}
	\item Signature-based detection
	\item Machine Learning (Classification)
	\item System redundancy
	\item Formal specification
\end{enumerate} 
 
Signature-based detection consists of a monitoring mechanism and compares it with pre-determined attack patterns known as attack signatures. This approach is popular among anti-virus corporations as well and relies on maintaining a database of signatures, which might curtail the usefulness of this approach. Machine learning-based approach monitors abnormal behaviors using a technique such as trained neural network or deep learning. The normal behavior is defined by supervised and unsupervised methods in the training phase. The disadvantage of this approach is the large training dataset that is needed to cover all the corner cases. Although unsupervised learning systems remove the need for labeled data, the issue of high false positives remain. Redundancy-based techniques duplicate key system components (such as the controller) and cross-check their states/outputs at runtime to detect intrusions/attacks/anomalies. The redundancy can be in hardware or it can be in software or a combination of both. However, from an attacker's standpoint, having redundant sensors increases the attack surface while increasing the attackers effort by only a constant effort. In effect, this seems like the wrong game to play with the attackers. Formal specification based attacks rely on program execution invariants to capture normal system operations. The end result is a state machine that can be used to detect anomalous program states and transitions to detect anomalies/attacks. A summary of attack detection methods used in UAVs is given in Table \ref{table:attack-detection-uav}.

\begin{table}[h]
\caption{Some attack detection techniques used in UAVs. There are four broad techniques used for detecting attacks on UAVs and other CPS-based systems.}
\begin{tabular}{ |p{3cm}|p{3cm}|}
 \hline
 \multicolumn{2}{|c|}{Attack Detection Methods used in UAVs} \\
 \hline
 Type of Strategy used & References \\
 \hline
 Signatures & \cite{gao2014cyber, kaur2013automatic}\\
 \hline
 Machine learning & \cite{abbaspour2016detection, chen2018learning, junejo2016behaviour, shen2014novel}\\
 \hline
 System redundancy& \cite{fei2018cross, frank1990fault, xu1987fuzzy, yoon2017virtualdrone}  \\
 \hline
 Formal specification & \cite{bak2011sandboxing, mitchell2014adaptive,  mitchell2015behavior, zimmer2010time}  \\
 \hline
\end{tabular}

\label{table:attack-detection-uav}
\end{table}





\section{Background}
Control Systems which are Open-loop are rarely used today. Instead, what is popular today is known as Closed-loop systems which have a feedback element that is used to minimize the error between the desired mission and the actual mission being executed. This is necessary since it is impossible to perfectly model beforehand the vagaries of the external disturbances as well as the noise inherent in the system.

An example of a closed loop system is as shown below:

%Insert picture of a closed loop system here.
\input{experiments}

\section{Conclusions}


\appendix

%\section{Location}

%Note that in the new ACM style, the Appendices come before the References.

%\section{Related Work}
The idea of using models to characterize normal behavior has been tried across a wide spectrum of systems ranging from industrial control plants \cite{wang2014srid}, capturing the physics of sensors \cite{shoukry2015pycra}, medical devices \cite{hei2013pipac}, and other control systems \cite{mclaughlin2013cps}. The equivalent of a model in control theory is called \textit{state estimation}. The central idea is to characterize the variations in key parameters as well as the noise that is inherent in the physical environment and the measurement. Models typically capture a universal law in the domain of application. For instance, in industrial control plants, laws of fluid dynamics such as Bernoulli's equations can be used to capture the model. However, as the applications of models have varied wildly from one another, there has not been enough discussion to provoke the use of model-based anomaly detection in a general sense. \\

\subsection{Security \& Privacy in CPS} Security and privacy of CPS systems is a well-researched area. An example is is verification of control code by an embedded system before it reaches the Programmable Logic Controller(PLC), Remote Terminal Unit or Intelligent Electronic Device(IED) \cite{mclaughlin2014trusted}. This work came out in the context of the popular Stuxnet \cite{falliere2011w32} security breach and triggered a bunch of other works in this space. Examples being security of embedded devices \cite{lemay2009cumulative}, the automatic generation of malicious PLC payloads \cite{mclaughlin2012sabot}, medical device security \cite{rushanan2014sok}, vulnerability analysis of vehicles \cite{checkoway2011comprehensive}, and of automated meter readings \cite{rouf2012neighborhood}. There is work on CPS privacy such as smart grids \cite{jawurek2012sok}, vehicular location monitoring \cite{hoh2008virtual}, and location privacy \cite{shokri2011quantifying}. These are related but complementary to the problem that we are describing in this paper. 

Instead, in this work,we are going to be focusing on using measurements of the physical world via sensors to build up an indicator of attacks. This work is inspired by the work on false sensor measurements \cite{liu2011false}, or false control signals such as manipulating vehicle platoons \cite{gerdes2013cps}, manipulating demand-response systems \cite{tan2013impact}. The question we ask is this: \textit{How do we raise a flag when something is wrong with sensors or actuators?}. Intrusion detection is a problem that is closely related to this. A classic paper which considers intrusion detection in industrial control networks is Cheung et al. \cite{cheung2007using}. It is noted in this paper that the technique of network anomaly detection is more effective in bounded, finite control networks where network flows are more regular and stable compared to the traditional computer networks used in the internet.

\subsubsection{Secure System Identification} One of the main areas of research in the CPS community is to find efficiently the subset of sensors that are sending false information \cite{chong2015observability}. These systems are assumed to satisfy the equations for Linear, Time-invariant systems which has been noted earlier. The main idea here is to solve a combinatorial optimization problem to find a subset of sensors, which are assumed to be safe, to generate adequate state estimations. Shoukry et. al \cite{shoukry2018smt} proposed a search algorithm based on Satisfiability Modulo Theory(SMT) to speed up the search of possible sensor sets and extended this to systems subject to random noise \cite{mishra2017secure}. In particular, it was noted that the number of sensors used to monitor the process has to be atleast twice the number of sensors under attack. In this paper, the authors  assume that the controllers and actuators can be trusted. Furthermore, there is a degree of hardware redundancy assumed in this paper which might not be always true in practical systems

\subsection{Attacks on UAVs} Attacks on UAVs do not surprise anymore. In \cite{checkoway2011comprehensive, koscher2010experimental}, the internal vehicular networks are infiltrated by subverting the CAN bus. CAN bus, being a broadcast-based protocol is shown to be subverted by an adversary with a laptop having a wireless card. In \cite{ishtiaq2010security}, the authors exploit a car tire pressure sensing system, which uses RF wireless motes. Kerns et. al \cite{kerns2014unmanned} consider how the Global Positioning System (GPS) signals can be used to take over UAVs. The attacker generates a counterfeit GPS signal and sends it to the GPS antenna of the UAV. This replaces the real GPS reading with a fake reading. The paper also proposes a detection strategy by modeling the UAV's state and using a \textit{residual test} (a system that takes in sensor readings outputs a signal called the residual as per a mathematical formulation). This paper is interesting and shows 2 attacks -- one where the attacker is detected, and another where the attacker manages to keep all the residuals below the threshold while still steering the aircraft to their location. Optical sensor input spoofing \cite{davidson2016controlling} involves obtaining an implicit control channel by tricking optical flow sensors with a fake ground plane. In \cite{son2015rocking}, the authors propose using a gyroscopic sensor attack with intentional acoustic noise to crash drones. Furthermore, the authors of \cite{trippel2017walnut} compromise accelerometers by injecting acoustic noise in a more controlled manner, as a more advanced form of attack. Anti-lock Braking System(ABS) attack \cite{shoukry2013non} involves injecting magnetic field to spoof the wheel speed sensor. In \cite{highnam2016uncrewed}, an attacker with an antenna and malicious ground station can compromise a benign UAV by sending malicious packets. In \cite{petit2015remote}, there are attacks on camera-based ground vehicles by relaying and spoofing signals.


\subsubsection{Detecting Attacks on UAVs} Detecting attacks on UAVs has been explored by multiple angles. The approaches tried out by the community can be classified into four buckets. Note that all of these approaches can be broadly called as \textit{Model-based} approaches.

\begin{enumerate}
	\item Signature-based detection
	\item Machine Learning (Classification)
	\item System redundancy
	\item Formal specification
\end{enumerate} 
 
Signature-based detection consists of a monitoring mechanism and compares it with pre-determined attack patterns known as attack signatures. This approach is popular among anti-virus corporations as well and relies on maintaining a database of signatures, which might curtail the usefulness of this approach. Machine learning-based approach monitors abnormal behaviors using a technique such as trained neural network or deep learning. The normal behavior is defined by supervised and unsupervised methods in the training phase. The disadvantage of this approach is the large training dataset that is needed to cover all the corner cases. Although unsupervised learning systems remove the need for labeled data, the issue of high false positives remain. Redundancy-based techniques duplicate key system components (such as the controller) and cross-check their states/outputs at runtime to detect intrusions/attacks/anomalies. The redundancy can be in hardware or it can be in software or a combination of both. However, from an attacker's standpoint, having redundant sensors increases the attack surface while increasing the attackers effort by only a constant effort. In effect, this seems like the wrong game to play with the attackers. Formal specification based attacks rely on program execution invariants to capture normal system operations. The end result is a state machine that can be used to detect anomalous program states and transitions to detect anomalies/attacks. A summary of attack detection methods used in UAVs is given in Table \ref{table:attack-detection-uav}.

\begin{table}[h]
\caption{Some attack detection techniques used in UAVs. There are four broad techniques used for detecting attacks on UAVs and other CPS-based systems.}
\begin{tabular}{ |p{3cm}|p{3cm}|}
 \hline
 \multicolumn{2}{|c|}{Attack Detection Methods used in UAVs} \\
 \hline
 Type of Strategy used & References \\
 \hline
 Signatures & \cite{gao2014cyber, kaur2013automatic}\\
 \hline
 Machine learning & \cite{abbaspour2016detection, chen2018learning, junejo2016behaviour, shen2014novel}\\
 \hline
 System redundancy& \cite{fei2018cross, frank1990fault, xu1987fuzzy, yoon2017virtualdrone}  \\
 \hline
 Formal specification & \cite{bak2011sandboxing, mitchell2014adaptive,  mitchell2015behavior, zimmer2010time}  \\
 \hline
\end{tabular}

\label{table:attack-detection-uav}
\end{table}






%\begin{acks}
% TODO: For the submission, don't include acknowledgments since they would most likely deanonymize you.
%\end{acks}
