\section{Security Challenge} Having covered the basics of SLAM and looked at the solutions used in practice, let us revisit the attacker threat model. The attacker is assumed to compromise the sensor readings by generating false sensor values and spoofing the sensor values to the controller. The attacker can accomplish this by subverting the sensing mechanism itself (such as using loud acoustic noise to compromise the on-board sensor) or by performing a Man-in-the-middle (MitM) attack on the channel connecting the sensors and controller (via a Ground Station).  The mode of the attacker, however, is irrelevant to this problem. Being a feedback-driven control loop, the sensor values then circulate within the system leading to huge estimation errors in the Extended Kalman Filtering (EKF) used to perform Simultaneous Localization and Mapping (SLAM). Our goal in this paper is to protect SLAM from sensor spoofing attacks and detect these attacks. The mitigation strategy for this is out of the scope of the paper but it can easily be something as simple as a return-to-home (RTH) functionality that is common in commercial drones today \cite{parrot_blog_2018}.